\newcommand{\actionx}{\kw{ACTIONX}}
\chapter{Programming in the deck: \actionx}
\label{actionx}
The \actionx{} keyword is the most direct way to \emph{program} in the deck. The
\actionx{} functionality consist of the \actionx{} keyword itself, with some
metadata and a condition and then a list of keywords which are injected into the
in-memory representation of the \kw{SCHEDULE} section at the point in time where
the condition evaluates to true. The \actionx{} statement is evaluated at the
end of every timestep, and if it evaluates to true the new keywords should take
effect immediately. The \actionx{} conditions are less sophisticated than the
expressions used in \udq{}, this implies that a common pattern is to make
involved calculations as \udq{} expressions, and then use a simple test as
\actionx{} condition.

The \actionx{} keyword is also documented in section 12.3.6 in the \flow{}
reference manual.

\section{Structure of the \actionx{} keyword}
\label{actionx_structure}
The \actionx{} keyword itself consist of multiple records. The first record is
metadata with name of the action, the number of times the action can be
triggered and the minimum time elapsed before an action is eligible for a second
run. The subsequent records are \emph{conditions}, all the conditions are of the
same form
\begin{code}
  lhs comparison rhs
\end{code}
and subsequent conditions are combined with \inlinecode{AND} or \inlinecode{OR}.
The \inlinecode{lhs} is a field, well or group quantity, in addition you can use
time variables \kw{DAY}, \kw{MNTH} and \kw{YEAR} as left hand
side\footnote{\eclipse{} supports a wider list of summary variables like region,
block and aquifer quantities on both left and right hand side.}. As with the
\udq{} variables the well and group variables are \emph{sets}, and the
evaluation status is maintained individually for each well and group.

The comparison is one of the ordinary mathematical comparison operators
\inlinecode{>,<,=,!=, <=} and \inlinecode{>=}. Numerical comparisons are done
with the corresponding plain C++ operators, this is in contrast to the \udq{}
implementation where an epsilon defined in \kw{UDQPARAMS} is used in floating
point comparisons.

The \inlinecode{rhs} is a numerical scalar, or a field, well or group quantity.
If your \inlinecode{rhs} is a well or group quantity the \inlinecode{lhs} and
\inlinecode{rhs} must be of the same type. If you use the symbol \kw{MNTH} as
\inlinecode{lhs} you can compare with named months, i.e. the following will
trigger on leap days
\begin{deck}
ACTIONX
  LEAP 1000 /
  MNTH=FEB AND /
  DAY=29 /
/
...
...
ENDACTIO
\end{deck}
When there is a well/group quantity as \inlinecode{lhs} the evaluation status is
maintained individually for each well/group. The complete condition evaluates to
true if \emph{any} of the wells/groups satisfy the condition. In the case of
wells the wells matching the condition can subsequently be accessed with
wellname '?' in the \kw{ACTIONX} keywords, this is a quite common pattern to
e.g. close the well with highest watercut.

If there are more conditions they must be joined with a trailing \kw{AND} or
\kw{OR}, furthermore conditions can be grouped with paranthesis. The \actionx{}
expressions can only contain the four arithmetic operators $+,-,*,/$ and not
mathematical functions like $\log()$, for more advanced expressions the natural
approach is to first define a \udq{} and then use the \udq{} symbol in the
\actionx{}, this is illustrated in section \ref{uda}. When multiple conditions
involving the same well set are evaluated, the list of matching wells available
in '?' will contain all the wells from the final condition, i.e. for
\begin{code}
 WWCT = {"OP1": 0.25, "OP2": 0.50, "OP3": 0.75}
\end{code}
and the action
\begin{deck}
ACTIONX
  WWCT /
  WWCT > 0.33 AND /
  WWCT < 0.66 /
/
...
ENDACTIO
\end{deck}
the set of wells available for further use in '?' are \emph{all} the wells
matching the condition \inlinecode{WWCT < 0.66} i.e. OP1 and OP2 and \emph{not}
the wells matching the combined expression \inlinecode{0.33 < WWCT < 0.66}. In
order to select wells in a range as attempted here you will have to create an
indicator variable with \udq{} first and then select based on that indicator -
e.g. something like
\begin{deck}
UDQ
  DEFINE WUCTR (WWCT < 0.66) * (WWCT > 0.33) /
/

ACTIONX
  WUCTR /
  WUCTR = 1 /
/

...

ENDACTIO
\end{deck}

The \actionx{} implementation is located in
\path{opm/input/eclipse/Schedule/Action} and all the classes are in namespace
\inlinecode{Action::}. As with the \udq{} the input parser needs some special
case to handle '/' and '*' as division operator and multiplier respectively, but
that is the only code shared between the \udq{} and the \actionx{}
implementation\footnote{It might be possible to share more code between the two,
in particular both have an internal recursive descent parser, but both \udq{}
and \actionx{} have so much ``personality'' that at least initially separate
implementations was the simplest.}.

The condition part of the \actionx{} keyword is internalized while the
\kw{SCHEDULE} section is parsed, the final product is maintained in a class
\inlinecode{Action::ActionX} which has a \inlinecode{eval()} method waiting to
be called. The keywords in the \actionx{} block are stored in the
\inlinecode{Action::ActionX} keyword for future use. All of the \actionx{}
keywords are stored in a container \inlinecode{Action::Actions} which will
eventually manage the book keeping of which actions are eligible for evaluation.

\section{The structure of the \inlinecode{Schedule} implementation}
\label{schedule_design}
\flow{} internalizes all keywords from the input deck and passes fully baked
datastructures to the simulator, whereas our impression is that \eclipse{} works
more like a reservoir model interepreter, executing one keywords at a time.
Mostly the \flow{} approach has worked out well, however for the \actionx{}
functionality the difference in execution model is quite acute, and the nature
of the \actionx{} keyword has had quite strong influence on the final Schedule
implementation. Although not required for use of \actionx{} it is valuable to
understand how the \actionx{} functionality has influenced the design of the
Schedule class, that way you will hopefully better understand problems or bugs
which might arise in the future.

At the very first pass the \kw{SCHEDULE} section is split in \emph{blocks}, with
one block for each report step. The blocks are implemented with the class
\inlinecode{ScheduleBlock}. Each block has a starting time and a list of
keywords, the keywords are maintained in the input format
\inlinecode{DeckKeyword}. Then the entire \kw{SCHEDULE} section is internalized
in the class \inlinecode{ScheduleDeck} which essentially contains a list of
\inlinecode{ScheduleBlock} instances. Consider the \kw{SCHEDULE} section
\begin{deck}
START
  1 'JAN' 2020 /

...
...

SCHEDULE

WELSPECS
	'PROD'	'G1'	10	10	8400	'OIL' /
	'INJ'	'G1'	1	1	8335	'GAS' /
/

COMPDAT
	'PROD'	10	10	3	3	'OPEN'	1*	1*	0.5 /
	'INJ'	1	1	1	1	'OPEN'	1*	1*	0.5 /
/

-- End of block 0

DATES
   1 'FEB' 2020 /
/

WCONPROD
	'PROD' 'OPEN' 'ORAT' 20000 4* 1000 /
/

WCONINJE
	'INJ'	'GAS'	'OPEN'	'RATE'	100000 1* 9014 /
/

-- End of block 1

DATES
  1 'MAR' 2020 /
/

-- End of block 2
END

\end{deck}

When this is internalized we get a \inlinecode{ScheduleDeck} instance with three
\inlinecode{ScheduleBlock} values:

\begin{code}
ScheduleDeck sched_deck = [
  ScheduleBlock {
    start = "2020-01-01",
    keywords = ["WELSPECS","COMPDAT"]
  },
  ScheduleBlock {
    start = "2020-02-01",
    keywords = ["WCONPROD","WCONINJE"]
  },
  ScheduleBlock {
    start = "2020-03-01",
    keywords = []
  }
]
\end{code}

The \inlinecode{Schedule} class has a \inlinecode{ScheduleDeck} member. The
processed content of the \inlinecode{Schedule} class is managed in vector of
\inlinecode{ScheduleState} instances, where one \inlinecode{ScheduleState}
represents the complete dynamic input state at a particular report step. The
processed \kw{SCHEDULE} code is created with the method
\inlinecode{Schedule::iterateScheduleSection()}.

When \inlinecode{Schedule::iterateScheduleSection(report\_step)} is called it
starts by clearing the vector of \inlinecode{ScheduleState} instances from
\inlinecode{report\_step} to the end of the simulation, and then recreates those
by treating the \inlinecode{ScheduleBlock}. The advantage of this approach is
that the \inlinecode{Schedule::iterateScheduleSection(report\_step)} method is
idempotent - it can be called repeatedly, from an arbitrary point in the
timeseries.

The implementation of the \actionx{} functionality is just to append the
\actionx{} keywords in the \inlinecode{ScheduleBlock} instance corresponding to
the current report step and then rerun the
\inlinecode{Schedule::iterateScheduleSection()} from this report step.



\section{Forward references of wells and groups}
When a well or group is defined as an \actionx{} keyword and then
unconditionally referenced in the deck we get a challenge at the first pass
through the \kw{SCHEDULE} section. In the example below a new well \kw{W1} is
defined with the \kw{WELSPECS} keyword when the action \inlinecode{NEW\_WELL}
evaluates to true. At 1.st of January 2025 the well \kw{W1} is opened with the
\kw{WCONPROD} keyword. The enginer making this model assumes that the
\inlinecode{NEW\_WELL} action will evaluate to true sometime before 1.st of
January 2025, and thereby ensure that the well is fully defined when it is
eventually opened with \kw{WCONPROD}:
\begin{deck}
ACTIONX
   'NEW_WELL/
   WWCT OPX  > 0.75 /
/

WELSPECS
  'W1'  'OP'  1 1 3.33  'OIL' 7*/
/

ENDACTIO

TSTEP
  10*30 /

DATES
   1 'JAN' 2025 /
/

WCONPROD
 'W1'      'OPEN'      'ORAT'      0.000      0.000      0.000  5* /
/

TSTEP
  10*30 /

\end{deck}

For \flow{} this creates problems because the entire \kw{SCHEDULE} section is
parsed when the simulator starts, and at first pass the well \kw{W1} is unknown
in the \kw{WCONPROD} keyword. This is ``solved'' in the following way:
\begin{enumerate}
\item At first pass we inspect the keywords inside the \actionx{} block and if
  we discover \kw{WELSPECS} we store the name of the well which will be defined
  at a later stage through \actionx{}.
\item When we parse further on as part of the first pass and said well is
  referenced e.g. in a \kw{WCONPROD} keyword, we verify that the well will
  eventually appear via \actionx{} - we issue a warning and ignore the well in
  the \kw{WCONPROD} keyword\footnote{If the well is not registered as ``will
  appear through \actionx{}'' there will be a runtime error with unknown well
  name when parsing \kw{WCONPROD}.}.
\item When the \actionx{} evaluates to true the well will be properly defined,
  and when reiterating over the Schedule keywords the \kw{WCONPROD} keyword will
  now be properly internalized. If the \actionx{} never evaluates to true the
  \kw{WCONPROD} keyword will never be applied, and the warning from point 2 will
  be the only trace of this well.
\end{enumerate}
It should be mentioned that the functionality with forward referencing of well
names is quite new\footnote{In January 2022}, there might be well keywords in
the \kw{SCHEDULE} section where the implementation is not yet prepared for this.
Furthermore the forward referencing is not at all implemented for groups. The
relevant data structure is the member \inlinecode{Action::WGNames
  action\_wgnames} in the \inlinecode{Schedule} class.


\section{To enable a new keyword for \actionx}
The keywords must be explicitly enabled to be available in an \actionx{} block,
and enabling a new keyword requires recompiling \flow{}. The keywords available
as \actionx{} keywords are listed in the static method
\inlinecode{ActionX::valid\_keyword()} in
\path{opm/input/eclipse/Schedule/Action/ActionX.cpp}. In principle it should
just be to add the keyword to the \inlinecode{ActionX::valid\_keyword()} method
and rebuild \flow{}, but experience has unfortunately shown that problems of
various kinds have had a tendency to pop up when new keywords are tried out as
\actionx{} keywords. Most commonly the problems have been in the interaction
between the \inlinecode{Schedule} class in opm-common and the simulator - things
have a tendency to go out of sync.


\section{Running \actionx{} during simulation}
The first part of the \actionx{} treatment is parsing the keyword and conditions
and assemble a syntax tree which can be used to evaluate the conditions. This
parsing takes place when the \kw{SCHEDULE} section is parsed for the first time.
This takes place fully within the realms of the opm-common codebase, and is
quite mature.

When the simulation actually runs the \actionx{} behavior consists of three
distinct parts, taking place in the simulator, in opm-common and again in the
simulator. The simulator will manage an instance of \inlinecode{Action::State}
which will hold on to the time of last run and the latest results for the
various actions.

\subsection{Prepare and evaluate}
As with \udq{} the \inlinecode{SummaryState} instance is the most important
variable to provide context to the \actionx{} evaluation, i.e. the
\inlinecode{SummaryState} variable must be evaluated before \actionx{}. In
addition the \udq{} variables must be evaluated and available in the
\inlinecode{SummaryState} instance before we invoke the \actionx{}
functionality.

The evaluation of actions is called from the method \inlinecode{applyActions()}
in \path{eclproblem.hh}. The method will evaluate which actions are eligible for
running by inspecting the \inlinecode{Action::Actions} variable and call the
\inlinecode{ActionX::eval()} method.

\subsection{Recreate \inlinecode{Schedule}}
When an \actionx{} has evaluated to \inlinecode{true} the simulator will call
into the opm-common method \inlinecode{Schedule::applyAction(report\_step)}.
That function will add the keywords from the \actionx{} keyword to the
\inlinecode{ScheduleBlock} for the correct report step, and then reiterate
through the \kw{SCHEDULE} section to the end of the simulation.

This reiterate process will recreate all internal members in the
\inlinecode{Schedule} class, i.e. if the simulator was holding on to a reference
to an internal \inlinecode{Schedule} datastructure that will be invalidated.

While recreating the \inlinecode{Schedule} instance there is some book keeping
as to which datastructures need to be recalculated in the simulator as a
consequence of the \actionx{}. That information is maintained in the data
structure \inlinecode{Action::SimulatorUpdate} which is returned back to the
simulator.


\subsection{Updating simulator data structures}
The \actionx{} implementation is in the module opm-common, whereas when the
simulation is proceeding it is the simulator code which is clearly in control.
If an action has evaluated to true and new keywords are injected in the Schedule
object the complete simulator state is updated. This is complex, and many of the
bugs in \actionx{} functionality have been in the interaction between the
simulator and opm-common, in particular when an action has evaluated to true.

An assumption permeating the simulator code is that changes to the well and
group configuration only take place at report steps, and in between those the
simulator ``owns'' the well and group data. Unfortunately this is no longer the
case when \actionx{} is active, and depending on the keywords in the \actionx{}
block we need to update the simulator data structures after \actionx{} has been
evaluated to true. Some details of what is currently updated is decsribed below.
This update mechanism will probably need to be continously updated in the
future.

The simulator code makes \emph{copies} of many of the objects like wells and
connections from the \inlinecode{Schedule} class, and also assembles many
simulator specific data structures which to a large extent consist of extracts
of information from the internals of the \inlinecode{Schedule} object. Some of
the interaction between the simulator and the input layer could probably be
simplified if the simulator would call the \inlinecode{Schedule} object when
e.g. a well or connection is needed, instead of storing references or copies to
the \inlinecode{Schedule} objects internally.

There are currently three categories of changes that can take place due to
\actionx{}:


\subsubsection{General changes in well status}
\label{actionx_change_well}
When there is well related keyword in the \actionx{} block - e.g. \kw{WELOPEN}
to open or close a well or \kw{WCONPROD} to adjust rates, the simulator is
required update it's internal data structures with the updated input information.
This will be communicated by flagging all the wells which need an update in the
\inlinecode{Action::SimulatorUpdate} instance which is passed from the
simulator.


\subsubsection{Changes in geo properties}
The grid keywords are in general not permitted in the \kw{SCHEDULE} section,
however there are a few geo multipliers like \kw{MULTZ} and \kw{MULTFLT} which
are allowed in the \kw{SCHEDULE} section, and thereby also in \actionx{}. If one
of these keywords are encountered in the \actionx{} block we set the flag
\inlinecode{Action::SimulatorUpdate::tran\_update = true} to encourage the
simulator to recalculate the transmissibilities.


\subsubsection{WELPI}
\label{actionx_welpi}
The \kw{WELPI} keyword is quite complex from the outset, when it is included as
an \actionx{} keyword it gets even more complicated. In order to support
\kw{WELPI} in \actionx{} the simulator needs to inspect the \actionx{} keywords
before invoking them, and if \kw{WELPI} is included the PI values must be
assembled from the simulator and passed to the
\inlinecode{Schedule::applyAction()}. This is implemented and works, but it is
complex and the special treatment in order to support the combination \kw{WELPI}
+ \actionx{} is quite considerable.


\section{Problems in parallel}
\label{actionx_paralle}
The operation environment for \actionx{} is quite similar to \udq{} when it
comes to parallel behavior - see section \ref{udq_parallel}, in addition we
need to be aware of parallel challenges after the \actionx{} has completed. If
the \actionx{} keyword changes the well structure there will be problems with
with the parallel well distribution in the simulator. \emph{As of Januarry 2022
this will fail undetected.}


\section{\actionx{} restart output}
As with the \udq{} keyword some of the structure and complexity of the
\actionx{} datastructures are there primarily to enable \eclipse{} compatible
restart. Regarding restart of \actionx{} related data:
\begin{enumerate}
  \item The restart output contains the result of parsing an \actionx{}
    condition in an intermediate representation which has been reverse
    engineered, this is complex and might not be 100\% correct.
  \item When restarting \flow{} from an \eclipse{} formatted restart file the
    \actionx{} conditions are reparsed based on string data in the restart file
    and the intermediate representation mentioned in point 1 is not utilised.
\end{enumerate}
